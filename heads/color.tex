%%%
%%%
%%%Farben die verwendet werden können:
%%% Primary Color:
%%%
%%% Farbe1 blue
\def\maincoloronenormal{66 42 173}
\def\maincoloronedark{41 27 107}
\def\maincoloroneultradark{16 10 42}
\def\maincoloronelight{121 99 217}
\def\maincoloroneultralight{184 173 235}
%%% Farbe2 brown
\def\maincolortwonormal{206 84 18}
%\def\maincolortwodark{113 46 10} Orginal-Farbe
\def\maincolortwodark{148 50 15}
\def\maincolortwoultradark{42 22 10}
\def\maincolortwolight{240 140 86}
\def\maincolortwoultralight{247 191 161}
%%%Farbe3 green 
\def\maincolorthreenormal{145 169 61}
\def\maincolorthreedark{85 101 27}
\def\maincolorthreeultradark{52 66 0}
\def\maincolorthreelight{195 218 114}
\def\maincolorthreeultralight{221 234 174}
%%%
%%%
%%%
%%%Secondary Color
\def\secondarycolorone{255 165 82}
\def\secondarycolortwo{200 162 6}
\def\secondarycolorthree{132 112 140}
\def\secondarycolorfour{211 199 149}
\def\secondarycolorfive{237 175 209}
\def\secondarycolorsix{159 175 144}
\def\secondarycolorseven{244 132 95}
\def\secondarycoloreight{233 215 88}
\def\secondarycolornine{110 110 110}



%%%      Allgemeine Farbdefinitionen; Überschriften und Dokumentübergreifendes. 
%%%     Im Moment wird green als Hauptthema verwendet. 
%%%



\definecolor{colorsection}{RGB}{\maincolorthreedark} %Für ÜBerschriften 
\definecolor{colorsubsection}{RGB}{\maincolorthreedark} %Für ÜBerschriften 
\definecolor{colorparagraph}{RGB}{\maincoloronedark} %Für ÜBerschriften 

%%%Kopf und Fuß
\definecolor{moohead}{RGB}{\maincolorthreeultradark}
\definecolor{moofoot}{RGB}{\maincolorthreeultradark}
\setkomafont{headsepline}{\color{moohead}}
\setkomafont{footsepline}{\color{moofoot}}

\addtokomafont{section}{\color{colorsection}} %Für ÜBerschriften
\addtokomafont{subsection}{\color{colorsubsection}} %Für ÜBerschriften
\addtokomafont{paragraph}{\color{colorparagraph}} %Für ÜBerschriften

\definecolor{mooparagraph}{RGB}{\maincoloronedark} 
%%%
%%% Highlight:
\definecolor{moohighlight}{RGB}{\maincoloronenormal}
\definecolor{mooimportant}{RGB}{\maincolortwonormal}
\definecolor{mooemphasize}{RGB}{\maincolorthreenormal}

%%% Host, NET, NEW-Net
\definecolor{moohost}{RGB}{\maincoloronenormal}  
\definecolor{moonet}{RGB}{\maincolortwodark}  
\definecolor{moonewnet}{RGB}{\maincolortwonormal}  


%%% Alice, Bob and Eve
\definecolor{mooalice}{RGB}{\maincoloronenormal}  
\definecolor{moobob}{RGB}{\maincolortwonormal}  

%%%% Aufzählende Erklärungen
\definecolor{moodescription}{RGB}{\maincolortwodark}
\definecolor{mooenumerate}{RGB}{\maincolortwodark}

%%% Hochzahlen %%%
\definecolor{moonumber}{RGB}{\maincoloronenormal}  %middlered
\definecolor{mooup}{RGB}{\maincolortwonormal}  %middlered
\definecolor{moobasis}{RGB}{\maincolorthreenormal}

%%%  Grau  %%%
\definecolor{mooiban}{RGB}{\secondarycolornine}


%%%%%% Spizalkonfig für IPv4 Header 
%\definecolor{mooup}{RGB}{255 110 110}
\definecolor{mooversion}{RGB}{\secondarycolorsix} 
\definecolor{moofragment}{RGB}{\secondarycolortwo} %lightpink
\definecolor{moottl}{RGB}{\secondarycolorthree} % lightyellow
\definecolor{mooprotocol}{RGB}{\secondarycolorfour} % lightblue
\definecolor{moochecksum}{RGB}{\secondarycolorfive} % lightorange
\definecolor{moosource}{RGB}{\secondarycolorone}  %lightlime
\definecolor{moodestination}{RGB}{\secondarycoloreight}  %lightturk
\definecolor{moobit}{RGB}{\maincolortwonormal}  %middlered

\newcommand\fadingtext[3][]{%
  \begin{tikzfadingfrompicture}[name=fading letter]
    \node[text=transparent!0,inner xsep=0pt,outer xsep=0pt,#1] {#3};
  \end{tikzfadingfrompicture}%

  \begin{tikzpicture}[baseline=(textnode.base)]
    \node[inner sep=0pt,outer sep=0pt,#1](textnode){\phantom{#3}}; 
    \shade[path fading=fading letter,#2,fit fading=false]
    (textnode.south west) rectangle (textnode.north east);% 
  \end{tikzpicture}% 
}




%%% 2 schön zueinander passende Farben:
% \definecolor{mooclientserver}{RGB}{255 100 0}  
% \definecolor{mooserverclient}{RGB}{0 180 200}  
