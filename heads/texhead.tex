\documentclass[
fontsize=11pt,
a4paper,
headsepline=true,
footsepline=true,
headinclude=false,
footinclude=false,
headings=small,
DIV=12
]{scrartcl}
%Entwicklung
\usepackage{layout} %Anzeigen der Dokumentenränder

%\usepackage{amsmath}
\usepackage{graphicx} % Bilder einfügen
\usepackage{tikz} % Zeichenprogramm
\usetikzlibrary{fadings}



\usepackage[english,main=ngerman]{babel} %Sprachen die geladenwerden
\usepackage{blindtext} %Dummytext

\usepackage[utf8]{inputenc} % Sonderzeichen im Text 
\usepackage{xcolor} % Farben
\usepackage{listings} % Code soll nicht verändert werden
\usepackage[autostyle=true]{csquotes} %Automatische Zitate
\usepackage[hidelinks]{hyperref} % Klickbare Links und Forms
%\usepackage{eforms}
%\usepackage{nameref} % extended links

\usepackage{verbatim} %Kommentare
\usepackage{bytefield} %% IPv4 and IPv6 Header
\usepackage{wrapfig} % Text um Grafik
\usepackage{multicol} % Mehrspaltig Tabellen Text ... 
%\usepackage{tabularx} %Tabellen breite angeben evtl. besser mit resize verkleinern?
\usepackage{colortbl} % Farben in Array Tabellen
\usepackage{booktabs}  % Tabellen horizontale linien
\usepackage{textcomp}  % Eurosymbole
\usepackage{paracol} % Mehrspaltige Dokumente

\usepackage{arydshln} % Arrys and Tabellen punkte und spalten verändern
\usepackage{nicematrix} %Schönere Matrizen
\usepackage{abraces} %Schönere Klammern

%\begin{comment}  Benötigt für Ref in Tabellen:
\makeatletter
\let\orgdescriptionlabel\descriptionlabel
\renewcommand*{\descriptionlabel}[1]{%
  \let\orglabel\label
  \let\label\@gobble
  \phantomsection
  \edef\@currentlabel{#1}%
  %\edef\@currentlabelname{#1}%
  \let\label\orglabel
  \orgdescriptionlabel{#1}%
}
\makeatother
%\end{comment}
%\usepackage{amssymb}
%\usepackage{amsfonts}
%\usepackage{mdsymbol}
\usepackage[T1]{fontenc}
\usepackage[light,condensed]{roboto}
\setkomafont{pageheadfoot}{\sffamily\roboto}
\usepackage[defaultsans,scale=0.95]{opensans} %Default Schriftart
\renewcommand{\seriesdefault}{l}
\renewcommand{\familydefault}{\sfdefault}
\usepackage[LGRgreek]{mathastext} %Damit die Matheschriftart auch opensans ist. 
\linespread{1.20}
\usepackage{tgcursor} %Typewriter Schriftart
\linespread{1.06}
\usepackage{wasysym} %Symbole wie smily

%Seiteneinstellungen für layout und Head Foot 
\usepackage{scrlayer-scrpage}
\ModifyLayer[addvoffset=-.6ex]{scrheadings.foot.above.line}
\pagestyle{scrheadings}
\voffset 2.5cm
\textheight 20cm
\RedeclareSectionCommand[
  beforeskip=-.8\baselineskip,
  afterskip=.5\baselineskip]{subsection}

%LAtex Fehler: Damit die Matrix Spalten auch mehr wie Standard werden können.
\setcounter{MaxMatrixCols}{20}
\newcommand{\colorbitbox}[3]{%
\rlap{\bitbox{#2}{\color{#1}\rule{\width}{\height}}}%
\bitbox{#2}{#3}}
\newcommand{\colorwordbox}[3]{%
\rlap{\wordbox{#2}{\color{#1}\rule{\width}{\height}}}%
\wordbox{#2}{#3}}

\setlength\columnsep{1cm}


\usepackage{mathtools}
